\makeglossaries

\newglossaryentry{statistik}
{
	name={statistik},
	description={teknik pengumpulan, pengolahan, penyajian serta penarikan kesimpulan dari suatu data yang digunakan dalam proses pengambilan keputusan}
}
\newglossaryentry{statistika}
{
	name={statistika},
	description={ilmu yang mempelajari teknik pengumpulan, pengolahan, penyajian serta penarikan kesimpulan dari suatu data yang digunakan dalam proses pengambilan keputusan}
}
\newglossaryentry{variabel}
{
	name={variabel},
	description={suatu karakteristik atau atribut yang dapat diasumsikan dengan nilai yang berbeda}
}
\newglossaryentry{populasi}
{
	name={populasi},
	description={kumpulan dari individu atau item yang mungkin diperoleh dari suatu pengamatan atau percobaan}
}
\newglossaryentry{sampel}
{
	name={sampel},
	description={bagian dari populasi yang diambil untuk keperluan analisis data}
}
\newglossaryentry{mean}
{
	name={mean},
	description={rata-rata atau jumlahan dari seluruh nilai observasi dibagi banyaknya obeservasi}
}
\newglossaryentry{modus}
{
	name={modus},
	description={data yang memiliki frekuensi terjadinya paling tinggi, data yang paling sering muncul}
}
\newglossaryentry{median}
{
	name={median},
	description={nilai yang terletak di tengah setelah data diurutkan}
}
\newglossaryentry{varians}
{
	name={varians},
	description={seberapa jauh perbedaan atau jarak setiap nilai observasi dalam suatu populasi dari rata-ratanya}
}
\newglossaryentry{standardeviasi}
{
	name={standar deviasi},
	description={akar kuadrat dari variansi}
}
\newglossaryentry{histogram}
{
	name={histogram},
	description={diagram batang yang digunakan untuk menyajikan data agar lebih mudah membaca dan memahami data}
}
\newglossaryentry{boxplot}
{
	name={boxplot},
	description={gambar yang digunakan untukmenjelaskan letak kuartil-kuartil, nilai maksimum, nilai minimum serta pencilan}
}
\newglossaryentry{korelasi}
{
	name={korelasi},
	description={metode statistika yang digunakan untuk menentukan ada atau tidaknya hubungan linear antarvariabel}
}
\newglossaryentry{koefisienkorelasi}
{
	name={koefisien korelasi},
	description={ukuran numerik untuk menentukan apakah dua atau lebih variabel terkait secara linear}
}