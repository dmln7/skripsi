\chapter{SIMPULAN DAN SARAN}
Bab ini memuat tiga subbab yaitu simpulan, implikasi, dan saran.

\section{Simpulan}
Simpulan harus pendek, merupakan deskripsi esensial, cenderung berbentuk pernyataan kualitatif, dan bukan angka-angka. Simpulan merupakan rangkuman dari jawaban pertanyaan penelitian atau hasil uji hipotesis dan sekaligus merupakan pemecahan permasalahan yang ada pada rumusan masalah

\section{Implikasi}
Implikasi adalah konsekuensi lebih lanjut dari temuan dalam simpulan. Biasanya implikasi menggunakan bahasa saran tetapi belum operasional.

\section{Saran}
Saran merupakan rekomendasi yang ditujukan kepada berbagai pihak terkait hasil penelitian dan menggunakan bahasa yang operasional. Implikasi dan saran harus sesuai dengan hasil penelitian yang telah terangkum dalam simpulan.